%!TEX program = xelatex
%!TEX TS-program = xelatex
%!TEX encoding = UTF-8 Unicode

\documentclass[12pt]{article} %这个我就不多说了,头文件
\usepackage{url} %这个我也不多说了
\usepackage{fontspec,xltxtra,xunicode} %最新的mactex都有

\defaultfontfeatures{Mapping=tex-text}
\setromanfont{Heiti SC} %设置中文字体
\XeTeXlinebreaklocale “zh”
\XeTeXlinebreakskip = 0pt plus 1pt minus 0.1pt %文章内中文自动换行,可以自行调节

\newfontfamily{\H}{Songti SC} %设定新的字体快捷命令
\newfontfamily{\E}{Weibei SC} %设定新的字体快捷命令
\begin{document}
\thispagestyle{empty}
\small{给一个比较简单的方法,在mac上折腾CJK有点麻烦,其实XeTeX就可以解决中文的问题。编码的改动其实不需要在mactex的设置里面改,写在前面然后注释掉就好了。\\
繁體字什麼的也是可以實現的。\\
当你需要打不同字体的时候,就需要用到这个\url{\newfontfamily},这样你可以在一行中显示多种字体。比如说:\\}
\Huge{{\H 宋体} {\E 魏碑} 黑体}
\end{document}